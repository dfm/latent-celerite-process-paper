% The layout of this document has been shamelessly stolen from the MIT licensed project:
% https://github.com/davidwhogg/GaussianProductRefactor
% by David W. Hogg, Adrian Price-Whelan, and Boris Leistedt

\documentclass[10pt]{article}

\usepackage{amsmath, bm, mathrsfs, amssymb}
\usepackage[dvipsnames]{xcolor}
\usepackage[hidelinks,
            colorlinks=true,
            linkcolor=NavyBlue,
            citecolor=darkgray,
            urlcolor=NavyBlue]{hyperref}
\usepackage{pifont}
\usepackage{graphicx}
\usepackage{doi}

\usepackage{natbib}
\bibliographystyle{dfm_abbrvnat}
\setcitestyle{round,citesep={,},aysep={}}

% text stuhh
\newcommand{\documentname}{\textsl{Note}}
\newcommand{\sectionname}{Section}
\newcommand{\equationname}{equation}
\newcommand{\notename}{Note}
\renewcommand{\tablename}{Table}
\newcommand{\acronym}[1]{{\small{#1}}}
\newcommand{\code}[1]{\texttt{#1}}
\newcommand{\foreign}[1]{\textsl{#1}}
\newcommand{\etal}{\foreign{et~al.}}

% math stuhh
\newcommand{\hquad}{~~}
\newcommand{\given}{\,|\,}
\newcommand{\dd}{\mathrm{d}}
\newcommand{\T}{^{\!\mathsf{T}\!}}
\newcommand{\inv}{^{-1}}
\newcommand{\scalar}[1]{#1}
\renewcommand{\vector}[1]{\boldsymbol{#1}}
\newcommand{\tensor}[1]{\mathbf{#1}}
\renewcommand{\matrix}[1]{\mathsf{#1}}
\newcommand{\normal}{\mathcal{N}\!\,}

% variables
\newcommand{\celerite}{\code{celerite}}

% Journal keys - kill me now
\newcommand{\aj}{Astron.\,J.}
\newcommand{\apj}{Astrophys.\,J.} % Astrophysical Journal
\newcommand{\apjs}{Astrophys.\,J.\,Supp.\,Ser.} % Astrophysical Journal Supplements

% page layout stuff
\setlength{\topmargin}{0.0in}
\setlength{\headheight}{0.0in}
\setlength{\headsep}{0.0in}
\setlength{\parindent}{\baselineskip}
\setlength{\textwidth}{4.3in}
\setlength{\textheight}{2\textwidth}
\raggedbottom\sloppy\sloppypar\frenchspacing

% this might be crazy, but here's my number
\setlength{\marginparsep}{0.15in}
\setlength{\marginparwidth}{2.7in}
\usepackage{marginfix} % necessary but possibly evil
\newcounter{marginnote}
\setcounter{marginnote}{0}
\renewcommand{\footnote}[1]{\refstepcounter{marginnote}\textsuperscript{\themarginnote}\marginpar{\color{darkgray}\raggedright\footnotesize\textsuperscript{\themarginnote}#1}}
\newcommand{\tfigurerule}{\rule{0pt}{1ex}\\ \rule{\marginparwidth}{0.5pt}\\ \rule{0pt}{0.25ex}}
\newcommand{\bfigurerule}{\rule{0pt}{0.25ex}\\ \rule{\marginparwidth}{0.5pt}\\ \rule{0pt}{1ex}}
\renewcommand{\caption}[1]{\parbox{\marginparwidth}{\footnotesize\refstepcounter{figure}\textbf{\figurename~\thefigure}: {#1}}}

% and make the left margin correct
\setlength{\oddsidemargin}{0.5\paperwidth}
\addtolength{\oddsidemargin}{-1.0in}
\addtolength{\oddsidemargin}{-0.5\textwidth}
\addtolength{\oddsidemargin}{-0.5\marginparwidth}
\addtolength{\oddsidemargin}{-0.5\marginparsep}

\begin{document}

\section*{Fast \& scalable latent Gaussian processes for multivariate time-domain astronomy\footnote{%
    The methods described here are available as part of the \href{https://github.com/exoplanet-dev/celerite2}{\code{exoplanet-dev/celerite2} project on GitHub} (CITE ZENODO) and the code used to make the figures for this paper are available as part of the \href{https://github.com/dfm/latent-celerite-process-paper}{\code{dfm/latent-celerite-process-paper} project on GitHub} (CITE ZENODO).
  }}

\noindent\textbf{Daniel Foreman-Mackey}\footnote{%
  The authors would like to thank the Astronomical Data Group at Flatiron for listening to every iteration of this project and for providing great feedback every step of the way.
  We would also like to thank
  Suzanne Aigrain (Oxford),
  Gregory Guida (Etsy),
  Christopher Stumm (Etsy),
  \emph{and others\ldots}
  for insightful conversations and other contributions.
}\\
{\footnotesize%
\textsl{Center for Computational Astrophysics, Flatiron Institute, New York, NY, USA}%
}

\medskip\noindent\textbf{Tyler Gordon}\\
{\footnotesize%
\textsl{Department of Astronomy, University of Washington, Seattle, WA, USA}%
}

\medskip\noindent\textbf{Eric Agol}\\
{\footnotesize%
\textsl{Department of Astronomy, University of Washington, Seattle, WA, USA}%
}

\medskip\noindent\textbf{Megan Bedell}\\
{\footnotesize%
\textsl{Center for Computational Astrophysics, Flatiron Institute, New York, NY, USA}%
}

\medskip\noindent\textbf{Final author list and order TBD}\\
{\footnotesize%
\textsl{Affiliations}\\
\textsl{More affiliations}%
}

\paragraph{Abstract:}

% Probably better to re-write focusing more on the latent models instead of story time.

Univariate Gaussian process models have become a staple in the time-domain astronomy toolkit, driven in large part by computationally efficient methods that use fast linear algebra for structured matrices or state-space representation and Kalman filters.
In particular, the \emph{celerite} method \citep{foreman-mackey2017} has now been widely used, especially in the context of light curve modeling for single band surveys like Kepler, K2, and TESS.
This method was recently generalized to applications like transit spectroscopy where multiple bands are observed at all times \citep{gordon2020}.
In this paper, we further generalize to a broader range of research areas by adding support for observations in different bands at different times.
This can be interpreted as a model where there are a set of underlying (latent) Gaussian processes and the observations are generated as linear combinations of these latent processes.
The applications of this method include---but are not limited to---transient and stellar astrophysics with surveys like Rubin/LSST, models for stellar and systematic variability in extreme precision radial velocity experiments, reverberation mapping, and pulsar timing.

\section{Introduction}

A paper. \citep{foreman-mackey2017,foreman-mackey2018,gordon2020}

\citep{loper2020,hu2020}

\section{Background}

\subsection{Univariate \& Kronecker celerite processes}

\subsection{Backpropagation \& automatic differentiation}

\subsection{Other stochastic models for multivariate time series}

\section{Details}

\subsection{Generative process}

\subsection{The math}

\subsection{The algorithm}

\subsection{Implementation details}

\section{Examples}

\subsection{Modeling recommendations}

\subsection{Photometric time series from Rubin}

\subsection{Transient science with Rubin}

\subsection{Stellar variability in extreme precision radial velocity}

\subsection{Reverberation mapping}

\section{Discussion}




% \marginpar{\tfigurerule\\
%   \includegraphics[width=\marginparwidth, trim=0ex 0ex 0ex 0.5in, clip]{oned.pdf}
%   \caption{The one-dimensional case: If the prior pdf and the likelihood are both
%     Gaussian in a single parameter, their product (and hence the posterior pdf) is
%     also Gaussian, with a narrower width (smaller variance) than
%     either the prior pdf or the likelihood.\label{fig:oned}}\\
%   \bfigurerule}


% Render the references
\clearpage\raggedright
\bibliography{refs}

\end{document}
